\documentclass{jsarticle}
\usepackage{amsmath, amssymb} % 数学記号のパッケージ
\usepackage{ascmac} % 囲み枠のパッケージ

% ハイパーリンクに関するパッケージとその設定
\usepackage[dvipdfmx]{hyperref}
\usepackage{pxjahyper}
\hypersetup{
 colorlinks=true
}

% 定理などの番号付けに関するパッケージとその設定
\usepackage{amsthm}
\theoremstyle{definition}
\newtheorem{defi}{定義}[section]
\newtheorem{prop}[defi]{命題}
\newtheorem{thm}[defi]{定理}
\newtheorem{supp}[defi]{補足}
\newtheorem{word}[defi]{用語}
\newtheorem{ex}[defi]{例}
\newtheorem{lem}[defi]{補題}
\newtheorem{cor}[defi]{系}

\renewcommand\proofname{\bf 証明} % proof環境の設定変更

\begin{document}

% 本文
主に線形代数の復習.

\begin{lem}
$V$を有限次元線形空間,$\dim V=n$とする.
$m\leq n$に対して$v_1,\cdots,v_m\in V$が線形独立ならば,
$v_1,\cdots,v_n$が$V$の基底となるような$v_{m+1},\cdots,v_n\in V$が存在する.
\end{lem}

\begin{proof}
$k=n-m$に関する帰納法を用いて証明する.
$k=0$のとき,$v_1,\cdots,v_m$で生成される$V$の部分空間を$W$とすると
$\dim W=\dim V$ゆえ$W=V$であるから,
$v_1,\cdots,v_n$は$V$の基底である.
$k=l$のときに結論が成り立つと仮定し,
$v_1,\cdots,v_m\in V$が線形独立であると仮定する.
$v\in V$が$v_1,\cdots,v_m$の線形結合で表せないとすると,
$v_1,\cdots,v_m,v$は線形独立である.
よって,帰納法の仮定により$v_1,\cdots,v_m,v,v_{m+2},\cdots,v_n$が
$V$の基底となるような$v_{m+2},\cdots,v_n\in V$が存在する.
すなわち,線形独立な$v_1,\cdots,v_m$に
$v,v_{m+2},\cdots,v_n$を付け加えれば$V$の基底となる.
\end{proof}

\begin{lem}
$V$を有限次元線形空間,$\dim V=n$とする.
$m\geq n$に対して$v_1,\cdots,v_m\in V$が$V$を生成するならば,
$v_{i_1},\cdots,v_{i_n}$が$V$の基底となるような$i_1,\cdots,i_n\in\{1,\cdots,m\}$が存在する.
\end{lem}

\begin{proof}
$k=m-n$に関する帰納法を用いて証明する.
$k=0$のとき,$v_1,\cdots,v_m$は線形独立である.
もし線形独立でないとすると,ある$1\leq i\leq m$が存在して
$v_i$は$v_1,\cdots,v_{i-1},v_{i+1},\cdots,v_m$の線形結合で表せる.
よって$V$は$v_1,\cdots,v_{i-1},v_{i+1},\cdots,v_m$によって生成されるため
$\dim V\leq m-1<n$であるが,これは矛盾である.
従って,$v_1,\cdots,v_m$は$V$の基底である.
$k=l$のときに結論が成り立つと仮定し,
$v_1,\cdots,v_{m+1}\in V$が$V$を生成するとする.
このとき$v_1,\cdots,v_{m+1}$は線形従属である(そうでないと$\dim V=m+1>n$となり矛盾)から,
ある$1\leq i\leq m+1$が存在して
$v_i$は$v_1,\cdots,v_{i-1},v_{i+1},\cdots,v_{m+1}$の線形結合で表せる.
すなわち$V$は$v_1,\cdots,v_{i-1},v_{i+1},\cdots,v_{m+1}$によって生成されるから,
帰納法の仮定により
$v_{i_1},\cdots,v_{i_n}$が$V$の基底となるような
$i_1,\cdots,i_n\in\{1,\cdots,m+1\}\setminus\{i\}\subseteq\{1,\cdots,m+1\}$が存在する.
\end{proof}

\begin{cor}
$V$を有限次元線形空間,$W,W'$を$V$の部分空間とする.
線形写像$f\colon W\to W'$は$V$上の線形写像$\tilde{f}\colon V\to V$に拡張できる.
すなわち,$w\in W$に対して$\tilde{f}(w)=f(w)$となるような
線形写像$\tilde{f}\colon V\to V$が一意的に存在する.
\end{cor}

\begin{proof}
$\dim V=n,\dim W=m$とする.
$W$の基底を$v_1,\cdots,v_m$とすると,補題0.1より
$v_1,\cdots,v_n$が$V$の基底となるような$v_{m+1},\cdots,v_n\in V$が存在する.
任意の$v=a_1v_1+\cdots+a_nv_n\in V$に対して
\[ \tilde{f}(v):=f(a_1v_1+\cdots+a_mv_m) \]
と定めれば,$\tilde{f}\colon V\to V$は
$w\in W$に対して$\tilde{f}(w)=f(w)$となるような線形写像である.
\end{proof}

\end{document}