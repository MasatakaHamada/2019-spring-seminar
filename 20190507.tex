\documentclass{jsarticle}
\usepackage{amsmath, amssymb} % 数学記号のパッケージ
\usepackage{ascmac} % 囲み枠のパッケージ

% ハイパーリンクに関するパッケージとその設定
\usepackage[dvipdfmx]{hyperref}
\usepackage{pxjahyper}
\hypersetup{
 colorlinks=true
}

% 定理などの番号付けに関するパッケージとその設定
\usepackage{amsthm}
\theoremstyle{definition}
\newtheorem{defi}{定義}[section]
\newtheorem{prop}[defi]{命題}
\newtheorem{thm}[defi]{定理}
\newtheorem{supp}[defi]{補足}
\newtheorem{word}[defi]{用語}
\newtheorem{ex}[defi]{例}
\newtheorem{lem}[defi]{補題}
\newtheorem{cor}[defi]{系}

\renewcommand\proofname{\bf 証明} % proof環境の設定変更

\begin{document}

% 本文
\section{Continuity of Brownian motion and the finite-dimensional distributions}

\begin{prop}
$\{\omega\in\Omega:t\mapsto B(t,\omega)は連続\}\notin
\sigma(B(t_1),\cdots,B(t_n):0\leq t_1\leq\cdots\leq t_n, n\in\mathbb{N})$
\end{prop}

\begin{proof}
以下
$\mathcal{F}:=\sigma(B(t_1),\cdots,B(t_n):0\leq t_1\leq\cdots\leq t_n, n\in\mathbb{N})$
とおく.
\begin{enumerate}
% Step.1
\item
$A\in\mathcal{F}\Leftrightarrow$
ある高々可算な$\Gamma\subset[0,\infty)$および$S\in\mathcal{B}(\mathbb{R}^\Gamma)$が存在して
$A=\{\omega\in\Omega:(B(t,\omega))_{t\in\Gamma}\in S\}$

($\Leftarrow$)

($\Rightarrow$)

% Step.2
\item
$A\in\mathcal{F}\Rightarrow$ある高々可算な$\Gamma\subset[0,\infty)$が存在して,
$\omega\in A,\omega'\in\Omega$に対し
$(B(t,\omega))_{t\in\Gamma}=(B(t,\omega'))_{t\in\Gamma}$
ならば$\omega'\in A$が成り立つ.

% Step.3
\item
$A:=C([0,\infty)\to\mathbb{R})$は2の性質を持たない.
\end{enumerate}
以上より
$A\in\mathcal{F}\Leftrightarrow$
ある高々可算な$\Gamma\subset[0,\infty)$および$S\in\mathcal{B}(\mathbb{R}^\Gamma)$が存在して
$A=\{\omega\in\Omega:(B(t,\omega))_{t\in\Gamma}\in S\}$
である.
\end{proof}

\section{An example of the stochastic process which is not a Brownian motion}
$\{B(t):t\geq0\}$をブラウン運動とし,
$U$を$\{B(t):t\geq0\}$と独立で$[0,1]$上の一様分布に従う確率変数とする.
このとき,独立性とFubiniの定理より
\[ \mathbb{P}[B(U)=0]
=\mathbb{P}\otimes\tilde{\mathbb{P}}[B(U(\tilde{\omega}),\omega)=0]
=\int_0^1\mathbb{P}[B(u,\omega)=0]\,du
=0 \]
である.

\section{Distribution of Gaussian random vectors}
特性関数を使ってみよう.

\section{Normal distribution and exponential distribution}
ガンマ分布を用いた証明があるかも?

\end{document}